\part{Brainstorming}

\begin{itemize}
    \item Enhanced Biomimicry: investigate other organisms with unique undulatory swimming patterns and incorporate those patterns into our design.
    \item Explore aspects of fish locomotion, such as lateral line sensing, and integrate them for improved efficiency.
    \item Advanced Fluid Dynamics Modeling: Refine the computational fluid dynamics (CFD) model by incorporating real-time environmental data, allowing for dynamic adjustments in response to changing conditions.
    \item Turbulence and water currents in the CFD model to simulate more realistic aquatic environments.
    \item Machine Learning Integration: Implement machine learning algorithms to optimize the undulatory swimming patterns based on real-time feedback from the environment.
    \item Train the system to adapt to different terrains, depths, or obstacles for enhanced autonomy.
    \item Sustainable Materials: exploring sustainable materials research for 3D printing the model, aligning with ecological considerations for underwater exploration.
    \item Multi-Agent Systems, investigating the feasibility of deploying multiple submersibles working collaboratively in a swarm, sharing information and optimizing propulsion collectively.
    \item Sensory integration, by integrating additional sensors, such as temperature or pressure sensors, to expand the submersible's data collection capabilities and environmental awareness.
    \item Energy harvesting which is the implementation of energy harvesting technologies, such as solar or hydrodynamic energy, to supplement the power source and extend the operational duration.
    \item Adaptive control strategies: Develop adaptive control strategies that allow the submersible to dynamically adjust its undulatory motion based on real-time sensory input, optimizing energy efficiency.
    \item Underwater communication: enhance communication capabilities for remote operation, possibly through acoustic or optical communication methods.
    \item Underwater Navigation and Mapping: Incorporate navigation algorithms to enable the submersible to autonomously navigate and map underwater environments, contributing to ocean exploration.
    \item Integration with Environmental Monitoring: Explore possibilities for integrating environmental sensors that contribute to scientific data collection, such as monitoring water quality, temperature, or marine life.
    \item Humanitarian and Environmental Applications: Explore applications in environmental conservation, such as monitoring coral reefs or underwater ecosystems, contributing to biodiversity preservation.
    \item Collaboration with Marine Biologists: Collaborate with marine biologists to gain insights into specific fish behaviors and optimize the submersible's design accordingly.
    \item Educational Outreach: Develop educational materials and outreach programs to engage students and the public in understanding the importance of biomimicry and robotics in ocean exploration.
    \item Miniaturization and Micro-Robotics: Investigate the feasibility of miniaturizing the design for applications in confined underwater spaces or for studying microorganisms.    
\end{itemize}
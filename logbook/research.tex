\research{Ricardo Guardado}{Journal Article}
\vspace*{-0.5cm}
\paragraph{APA Citation} \

\begin{hangparas}{.25in}{1}
Oliveira, R. (2013). \textit{Mind the Fish: Zebrafish as a Model in Cognitive Social Neuroscience}. Frontiers https://www.frontiersin.org/articles/10.3389/fncir.2013.00131/full
\end{hangparas}

\vspace*{-0.5cm}
\paragraph{Notes} \

\begin{itemize}
    \item Social competence in animals is key for their survival - there is a great need for behavioral flexibility in social interactions to navigate changes in the social environment.
    \item Social behavior, often times motivated from neurological activity, possesses impact on biological mechanisms
    \item A cognitive level of Analysis can be used in the interface between the brain and behavior to develop theories and hypotheses about social behavior. This allows for the testing of predictions. 
    \item Teleost fish as a model could be effective because of its closely related species and the availability of genetic tools for studying neural networks. 
    \item There is a need for a general appraisal mechanism that assesses the valence and salience of social stimuli across different sensory modalities and functional domains. 
    \item Optogenetic and transgenic techniques enable the monitoring of neural activity and experimental manipulations in Zebrafish larvae
    \item The integration of brain activity imaging with behavioral tasks in Zebrafish provides an opportunity to explore cognitive processes and their neural correlates.
\end{itemize}
    
\vspace*{-0.5cm}
\paragraph{Questions this source helped me answer} \

\begin{itemize}
    \item Why are Zebrafish used to study social neurocognition and behavior?
    \item What element in science can be used to study the linkage between neuroscience behavior and social behavior in environments. 
    \item What mechanisms have been used to help humans draw connections between neural and social activity 
\end{itemize}

\vspace*{-0.5cm}
\paragraph{Questions I have for the author} \ 

\begin{itemize}
    \item Is it possible to make computation models using neurological data alone?
    \item What models can be used for fish that have irregular movement patterns? For example, sharks can only swim in the background. 
    \item Social Neuroscience--field that investigates the neural mechanisms underlying social processes and behavior 
    \item Cognitive level of Analysis--focuses on understanding mental processes such as perception, memory, language, problem-solving, and decision-making. 
    \item Magnetic resonance imaging--uses powerful magnets, radio waves, and a computer to generate detailed images of the internal structures of the body.
\end{itemize}

\vspace*{-0.5cm}
\paragraph{Summary} \

The proposed conceptual framework involves recognizing key components of social competence, including social value and preferences. Zebrafish are a valuable model for comparative social neuroscience due to the diversity of social systems among closely related species and the availability of genetic tools. 

\vspace*{-0.5cm}
\paragraph{Implications for my project/thinking} \

To make a working computational model in an environment like the ocean, we can use neurological data collected from the locomotion of Zebrafish to make this model because of the benefits that using this model would provide. 

\newpage 

\research{Ricardo Guardado}{Research Article}
\vspace*{-0.5cm}
\paragraph{APA Citation} \

\begin{hangparas}{.25in}{1}
Beem K. (2020). \textit{Rolling in the Deep: Climate Change and Deep Sea Ecosystems}. Columbia University. https://climatesociety.ei.columbia.edu/news/rolling-deep-climate-change-and-deep-sea-ecosystems.
\end{hangparas}

\vspace*{-0.5cm}
\paragraph{Notes} \

\begin{itemize}
    \item Deep sea ecosystems play a crucial role in oceanic-planetary regulatory systems. Through deep ocean mixing, they help drive ocean currents and absorb heat and atmosphere CO2, acting a buffer against climate change. 
    \item Anthropogenic climate change poses specific threats to deep sea ecosystems, including ocean acidification, deep sea warming, and decreased availability of food. These impacts could affect the deep sea's ability to serve as a carbon and heat sink. 
    \item The health of deep sea ecosystems is closely connected to the health of surface water ecosystems. 
    \item Methane vent communities on the ocean floor house unique species of bacteria and worms adapted to feed on methane, preventing it from reaching the surface. This methane oxidation process captures a significant amount of methane annually, mitigating its impact on the atmosphere. 
    \item Ocean warming, induced by climate change, could lead to a climate-induced shift in warm currents, potentially releasing large amounts of frozen methane from the seafloor. 
    \item Despite the importance of the deep sea, there is a lack of policy and scientific information about this ecosystem, which constitutes 90\% of Earth's livable area.
    \item Anthropogenic--refers to processes, activities, or conditions that originate from or are caused by human actions, particularly those that have an impact on the environment or natural resources. 
    \item Methane vent communities--ecosystems that exist around deep-sea hydrothermal vents where methane is released from the ocean floor. 
    \item Induced climate change--Changes in the Earth's climate that are primarily caused or triggered by human activities.     
\end{itemize}
    
\vspace*{-0.5cm}
\paragraph{Questions this source helped me answer} \

\begin{itemize}
    \item What are some effects of climate change on the ocean floor and other deep parts of the ocean?
    \item How are ecosystems tied with ocean conditions?
    \item Can ecosystems be used as a means of analyzing how human activity is affecting the oceanic environment?  
\end{itemize}

\vspace*{-0.5cm}
\paragraph{Questions I have for the author} \ 

\begin{itemize}
    \item How feasible do you think it is to construct a moving robot that can reach deep parts of the ocean and take images and record other data such as temperature over long periods of time to see the effects?
    \item What type of data would give us valuable insights on the fact that the ocean floor and its environment is changing over time based on human activity?
\end{itemize}

\vspace*{-0.5cm}
\paragraph{Summary} \

The health of deep-sea ecosystems is closely connected to surface water ecosystems, and their ability to capture and store carbon is essential for mitigating climate change. The article highlights methane vent communities on the ocean floor, where unique species help prevent methane from reaching the atmosphere. 

\vspace*{-0.5cm}
\paragraph{Implications for my project/thinking} \

We can use this as an application to our computational model. If we were to make a model that is energy efficient and is able to use enhanced locomotion for movement, we could use our model to discover deep parts of the oceans that can read how they are being affected by climate change that is induced by human activity over long periods of time. 

\newpage

\research{Ricardo Guardado}{Journal Article}
\vspace*{-0.5cm}
\paragraph{APA Citation} \

\begin{hangparas}{.25in}{1}
Cregg, J., Kiehn O., Leiras, R. (2022). \textit{Brainstem Circuits for Locomotion}. Annual Reviews. https://www.annualreviews.org/doi/full/10.1146/annurev-neuro-082321-025137
\end{hangparas}

\vspace*{-0.5cm}
\paragraph{Notes} \

\begin{itemize}
    \item Midbrain nuclei, specifically the cuneiform nucleus play important roles in mediating the initiation and speed control of locomotion as well as selection of different gates 
    \item Activation of CnF-Vglut2 neurons initiates locomotion with a full range of speeds and gaits, while PPN-Vglut2 activation specifically induces slow alternating gaits.
    \item Discrepancies exist in the outcomes of studio on PPN-Vglut2 neurons, indicating potential variations in targeting approaches and stimulus paradigms. 
    \item Targeting PPN-Vglut2 neurons broadly can lead to variable effects, emphasizing the need for precise stimulation in different subpopulations for specific outcomes. 
    \item The study of brainstem circuits for locomotor control has progressed from classical methods to a combination of electrophysiological, molecular genetics, and behavioral approaches 
    \item Brainstem pathways involved in stopping and steering locomotion have been identified alongside those related to initiation and speed control. 
\end{itemize}
    
\vspace*{-0.5cm}
\paragraph{Questions this source helped me answer} \

\begin{itemize}
    \item What role do brain stem cells play in the locomotion of organisms?
    \item How complicated can computational neuroscience with brainstem signals become when it comes to studying very complex organisms?
    \item Is it practical to make a computational neuroscience model with organisms that have very extensive and complex neurosystems? 
\end{itemize}

\vspace*{-0.5cm}
\paragraph{Questions I have for the author} \ 

\begin{itemize}
    \item A lot of the experimentation was done on species that are not very complex neurologically. However, for more complex organisms like humans, what differences could be discovered in the finishing discussed in the article?
    \item Because you discovered brainstem pathways that were related in function, how easy is it to trace locomotion of activity when both pathways are at work? What are some things that distinguish one from the other? 
\end{itemize}

\vspace*{-0.5cm}
\paragraph{Summary} \

Downstream circuits transmitting MLR signals to the spinal cord have been identified, with parallel pathways for stopping and steering locomotion also recognized. Although the precise way in which  the locomotors behave is unknown, there is optimism about establishing a functional connectome from the brainstem to the spinal cord. This connectome would clarify how the locomotor system is implemented and how traits are controlled by command signals. 

\vspace*{-0.5cm}
\paragraph{Implications for my project/thinking} \

\begin{itemize}
    \item Trying to develop our own neurological data that record the synaptic activity of a certain species would not be practical as it requires extensive experimental setup and is difficult to track the activity of every neuron in opening and steering locomotion in the species. 
    \item It would be best for us to obtain our synaptic data from an online source that is publicly released so that we do not get into the issue of trying to obtain our own data 
    \item If we were to experiment, it would be best to use a species that has a reputation for being feasible to work with in terms of the complexity of their neurological system. For example, an easy species to work with are drosophila flies. 
\end{itemize}

\newpage

\research{Anish Goyal}{Journal Article}
\vspace*{-0.5cm}
\paragraph{APA Citation} \

\begin{hangparas}{.25in}{1}
Roussel, Y., Gaudreau, S. F., Kacer, E. R., Sengupta, M., \& Bui, T. V. (2021). \textit{Modeling spinal locomotor circuits for movements in developing Zebrafish}. eLife, 10, e67453. https://doi.org/10.7554/eLife.67453
\end{hangparas}

\vspace*{-0.5cm}
\paragraph{Notes} \

\begin{itemize}
    \item The study focuses on understanding the operation of spinal circuits dedicated to locomotor control in developing Zebrafish.
    \item Iterative models were built, coupling developing Zebrafish spinal circuits with simplified musculoskeletal models to replicate coiling and swimming movements.
    \item Neurons in the models were based on morphologically or genetically identified populations in the Zebrafish spinal cord.
    \item Simulations involved intact spinal circuits, circuits with silenced neurons, and circuits with altered synaptic transmission to explore the role of specific spinal neurons.
    \item Analysis of firing patterns and phase relationships aimed to uncover mechanisms behind locomotor movements in developing Zebrafish.
    \item Simulations demonstrated the transition between coiling and swimming and highlighted the importance of contralateral excitation to multiple tail beats.
    \item The models provided insights into the sensitivity of spinal locomotor networks to various factors, including motor command amplitude, synaptic weights, axon length, and firing behavior.
\end{itemize}
    
\vspace*{-0.5cm}
\paragraph{Questions this source helped me answer} \

\begin{itemize}
    \item What specific morphologically or genetically identified populations of neurons in the Zebrafish spinal cord were used in the models?
    \item How did the simulations demonstrate the transition between coiling and swimming, and what mechanisms were identified during this transition?    
\end{itemize}

\vspace*{-0.5cm}
\paragraph{Questions I have for the author} \ 

\begin{itemize}
    \item What were the key findings regarding the sensitivity of spinal locomotor networks to factors such as motor command amplitude, synaptic weights, axon length, and firing behavior?
\end{itemize}

\vspace{-0.5cm}
\paragraph{Vocabulary} \ 

\begin{itemize}
\item Spinal Circuits: Neural circuits or networks within the spinal cord responsible for specific functions, such as locomotor control.
\item Zebrafish: A small tropical fish commonly used in scientific research, particularly in the study of vertebrate development and genetics.
\item Musculoskeletal Models: Simplified models that represent the interaction between muscles and bones in the context of movement and locomotion.
\item Contralateral Excitation: Excitation of neurons on the opposite side of the body, common in neural circuits involved in coordinating movements on both sides.
\item Axon Length: The length of the axon, a part of a neuron that conducts electrical impulses away from the cell body.
\item Firing Behavior: The pattern and frequency of action potentials (nerve impulses) produced by neurons.
\item Coiling: A movement pattern characterized by the bending or winding of the body.
\item Synaptic Transmission: The process by which nerve impulses pass across the synapse (junction between neurons) to transmit signals.
Rhythm Generation: The generation of rhythmic patterns, often associated with repetitive movements such as locomotion.
\end{itemize}

\vspace*{-0.5cm}
\paragraph{Summary} \

Utilizing iterative models, the researchers coupled Zebrafish spinal circuits with simplified musculoskeletal models to replicate coiling and swimming movements. The models were constructed based on morphologically or genetically identified neuron populations in the Zebrafish spinal cord. Simulations included intact spinal circuits as well as circuits with silenced neurons or altered synaptic transmission. 

\vspace*{-0.5cm}
\paragraph{Implications for my project/thinking} \

The created open-source models from this research paper help to advance humanity's knowledge in spinal locomotion in fish. Such models can be used to form a baseline for analysis on energy efficiency in a project. 

\newpage

\research{Anish Goyal}{Website}
\vspace*{-0.5cm}
\paragraph{APA Citation} \

\begin{hangparas}{.25in}{1}
Oliver K. Ernst, Ph. D. (2020, July 9). \textit{Headless setup of Raspberry Pi--once and for all}. Medium. https://medium.com/practical-coding/headless-setup-of-raspberry-pi-once-and-for-all-de5a2c4f715b
\end{hangparas}

\vspace*{-0.5cm}
\paragraph{Notes} \

\begin{itemize}
    \item Use balenaEtcher to flash the SD card with the chosen image (a good alternative is RPi flasher)
    \item Set up Wi-Fi by enabling the NetworkManager daemon and editing /etc/wpasupplicant.conf
    \item Can enable SPI, I2C, UART, LCD screen, NFC, and Docker through the RPi configuration tool
    \item Devices that connect to the Raspberry Pi via SSH must share the same network as the Pi
\end{itemize}
    
\vspace*{-0.5cm}
\paragraph{Questions this source helped me answer} \

\begin{itemize}
    \item How can I set up a Raspberry Pi headlessly (without monitor/keyboard)?
    \item What are the essential steps for configuring WiFi and enabling SSH on a headless Raspberry Pi?
    \item How can I enable optional features like SPI, I2C, UART, and LCD via fbtft on a Raspberry Pi?    
\end{itemize}

\vspace*{-0.5cm}
\paragraph{Questions I have for the author} \ 

\begin{itemize}
    \item Can the guide be adapted for older Raspberry Pi models?
    What are the specific advantages of using a headless Raspberry Pi setup?
    \item Why did the author choose to use Raspberry Pi OS Lite (32-bit) in making the guide?
\end{itemize}

\vspace{-0.5cm}
\paragraph{Vocabulary} \ 

\begin{itemize}
\item Headless Setup: Configuring a device without the need for a monitor or keyboard.
\item SPI (Serial Peripheral Interface): A synchronous serial communication protocol used to transfer data between a master device and one or more peripheral devices.
\item I2C (Inter-Integrated Circuit): A communication protocol used to connect low-speed devices in a master-slave configuration.
\item UART (Universal Asynchronous Receiver-Transmitter): A hardware communication protocol used for serial communication between devices.
\item LCD (Liquid Crystal Display): A type of flat-panel display technology commonly used in devices like monitors and screens.
\end{itemize}

\vspace*{-0.5cm}
\paragraph{Summary} \

This source is a detailed guide for a headless setup of a Raspberry Pi Zero WiFi using Raspberry Pi OS Lite. It covers flashing the SD card, configuring WiFi, enabling SSH, and optional features such as SPI, I2C, UART, and LCD via fbtft.

\vspace*{-0.5cm}
\paragraph{Implications for my project/thinking} \

Learned how to set up a Raspberry Pi headlessly, which is valuable for this project and doesn't require extensive hardware setup. I also learned how to configure Wi-Fi and SSH integration between end devices.

\newpage

\research{Anish Goyal}{Journal Article}
\vspace*{-0.5cm}
\paragraph{APA Citation} \

\begin{hangparas}{.25in}{1}
Dukes, X., Littleton, J., Neville, T., Rollerson, C., \& Quinney, A. (n.d.). \textit{Object detection on Raspberry Pi}. American Society for Engineering Education. https://peer.asee.org/object-detection-on-raspberry-pi.pdf
\end{hangparas}

\vspace*{-0.5cm}
\paragraph{Notes} \

\begin{itemize}
    \item The source discusses the implementation of object detection on Raspberry Pi using machine learning models
    \item The project involves using a web camera, Raspberry Pi Kits (Model B+), and a Google USB accelerator to enhance detection speed.
    \item They implemented object detection using SSD-MobileNet for general objects and extend it to recognize weapons through transfer learning.
    \item Preliminary validation results indicate effective detection of general objects and certain weapons, but the detection speed for weapons needs improvement.
\end{itemize}
    
\vspace*{-0.5cm}
\paragraph{Questions this source helped me answer} \

\begin{itemize}
    \item How is object detection implemented on Raspberry Pi's using machine learning models with minimal resource cost?
    \item What are the key components involved in such a project, including hardware and software?
    \item What challenges are addressed in implementing an object detection toolchain?   
\end{itemize}

\vspace*{-0.5cm}
\paragraph{Questions I have for the author} \ 

\begin{itemize}
    \item What optimizations were considered in implementing the SSD-MobileNet model?
    \item Were there any bottlenecks in using a Raspberry Pi, such as poor memory/CPU performance?
\end{itemize}

\vspace{-0.5cm}
\paragraph{Vocabulary} \ 

\begin{itemize}
\item Transfer Learning: When a model trained on one task is adapted to a second related task.
\item Edge Devices: Devices located close to the source of data that process data locally rather than relying on cloud computing.
\end{itemize}

\vspace*{-0.5cm}
\paragraph{Summary} \

This source is a senior design project implementing object detection on Raspberry Pi using SSD-MobileNet to perform weapon detection through transfer learning. The project uses a Google USB accelerator to enhance detection speed and seamless software communication via the SSH (Secure Shell) protocol.

\vspace*{-0.5cm}
\paragraph{Implications for my project/thinking} \

The use of transfer learning can be applied to create a model for undulatory swimming by fine tuning certain parameters. A Raspberry Pi headless display is best-suited to meet this end.

\newpage

\research{Dobromir Iliev Iliev}{Research Article}
\vspace*{-0.5cm}
\paragraph{APA Citation} \

\begin{hangparas}{.25in}{1}
Thomas, A., Bates, K., Elchesen, A., Hartsock, I., Lu, H., \& Bubenik, P. (2021b, May 12). \textit{Topological data analysis of C. elegans locomotion and behavior}. Frontiers. https://www.frontiersin.org/articles/10.3389/frai.2021.668395/full
\end{hangparas}

\vspace*{-0.5cm}
\paragraph{Questions I have for the author} \ 

\begin{itemize}
    \item Could the author provide more details on the specific experimental setups and conditions used in the study?
    \item How does the synthesis of skeleton data through persistent homology contribute to a better understanding of stereotypical behaviors in C. elegans?
    \item Are there any limitations or challenges encountered in applying persistent homology to behavior analysis, and how were they addressed?
    \item What were the key findings regarding the sensitivity of spinal locomotor networks to factors such as motor command amplitude, synaptic weights, axon length, and firing behavior?
\end{itemize}

\vspace{-0.5cm}
\paragraph{Vocabulary} \ 

\begin{itemize}
\item Persistent Homology: The study of topological features in a dataset that persist across multiple scales.
\item Sliding Window Embeddings: Transforming time series data into point cloud data while preserving temporal information.
\item Microfluidic Devices: Devices designed to manipulate small amounts of fluid, used in confining C. elegans for experiments.
\item Viscosity: A measure of a fluid's resistance to flow.
\end{itemize}

\vspace*{-0.5cm}
\paragraph{Summary} \

The study applies persistent homology to analyze C. elegans behavior in different experimental conditions, distinguishing and classifying locomotion patterns. The method outperforms simpler approaches, providing insights into stereotypical behaviors and environmental impacts. The use of sliding window embeddings facilitates the detection of characteristic cycles in time series data, contributing to a quantitative summary of complex behaviors. 

\vspace*{-0.5cm}
\paragraph{Implications for my project/thinking} \

This source enhances my understanding of how persistent homology can be applied to behavior analysis, offering a potential method for quantifying and classifying patterns in my own research. The emphasis on distinguishing behaviors and the synthesis of skeleton data may inspire new approaches or considerations in my project. Additionally, it prompts me to explore the limitations and challenges associated with applying persistent homology to behavior studies.

\newpage

\research{Dobromir Iliev Iliev}{Research Article}
\vspace*{-0.5cm}
\paragraph{APA Citation} \

\begin{hangparas}{.25in}{1}
Saputra, A. A., Botzheim, J., \& Kubota, N. (2023, June 3). \textit{Neuro-cognitive locomotion with dynamic attention on topological structure}. MDPI. https://doi.org/10.3390/machines11060619
\end{hangparas}

\vspace*{-0.5cm}
\paragraph{Notes} \

\begin{itemize}
    \item The research proposes a novel concept in locomotion generation that integrates a cognitive model from an ecological psychology viewpoint.
    \item The goal is to decrease the gap between the cognitive model and the locomotion model, moving from a hierarchical to a parallel system structure.
    \item The attention controller is highlighted as a crucial starting process for handling exteroceptive sensory information required by further cognitive processes.
    \item It processes 3D point-cloud data as exteroceptive sensory information and controls the density of topological structures in localized areas for detailed object perception.
    \item A dynamic density topological map construction process, called DD-GNG, is introduced based on the Growing Neural Gas (GNG) algorithm.
    \item Compared to other topological reconstruction methods, DD-GNG is claimed to reduce processing time significantly (up to 70%).
    \item Affordance detection is emphasized as a critical element in integrating the relationship between attention and locomotion.
    \item The proposed affordance detector provides semantic function in locomotion behavior, offering object information in the context of the robot's current capabilities.
    \item Experimental trials in both simulation and real robot performance demonstrate the effectiveness of the proposed system in short-term adaptation to obstacles.
    \item The robot showcases the ability to change limb swinging patterns and foothold targets in response to sudden obstacles.
    \item The contributions of the research include the dynamic density topological map construction process, real integration between attention and affordance, and the development of a locomotion system that integrates external sensory information in short-term adaptation.
    \item The paper mentions the potential for future development in implementing the dynamic attention model, especially in mobile robot applications.
    \item The concept of neuro-cognitive locomotion is seen as having high prospects for achieving dynamic locomotion that integrates cognition with the locomotion generator.
\end{itemize}
    
\vspace*{-0.5cm}
\paragraph{Questions this source helped me answer} \

\begin{itemize}
    \item How does the proposed locomotion generation concept integrate cognitive models from an ecological psychology viewpoint?
    \item What role does the attention controller play in handling exteroceptive sensory information?
    \item What is the dynamic density topological map construction process, and how does it differ from other methods?
    \item How does the affordance detector contribute to the integration between attention and locomotion?
    \item What are the key findings and results from the experimental trials in simulation and real robot performance?   
\end{itemize}

\vspace*{-0.5cm}
\paragraph{Questions I have for the author} \ 

\begin{itemize}
    \item Can you provide more details on the specific parameters and settings used in the dynamic density topological map construction process (DD-GNG)?
    \item How do you envision further developing the locomotion generator part to consider more sensory feedback?
    \item Are there specific limitations or challenges encountered during the experiments that you plan to address in future research?
\end{itemize}

\vspace{-0.5cm}
\paragraph{Vocabulary} \ 

\begin{itemize}
\item Exteroceptive: Relating to stimuli that arise outside an organism.
\item Ecological Psychology: A field of psychology that studies the relationship between individuals and their environments.
\item Affordance: The potential actions or interactions that an individual perceives in their environment.
\item GNG (Growing Neural Gas): A neural network algorithm used for clustering and topological mapping.
\end{itemize}

\vspace*{-0.5cm}
\paragraph{Summary} \

The research introduces a novel concept in locomotion generation, integrating ecological psychology principles. The attention controller processes 3D point-cloud data, and a dynamic density topological map construction process (DD-GNG) efficiently clarifies details of objects. Affordance detection contributes to integrating attention and locomotion. Experimental trials demonstrate the system's effectiveness in short-term adaptation to obstacles.

\vspace*{-0.5cm}
\paragraph{Implications for my project/thinking} \

\begin{itemize}
    \item integrating principles from ecological psychology in the design of cognitive models for robot locomotion.
    \item Dynamic density topological maps for detailed object perception in the robot's surroundings.
    \item Affordance detection can enhance the adaptability of a robot to unforeseen obstacles.
\end{itemize}

\newpage

\research{Dobromir Iliev Iliev}{Research Article}
\vspace*{-0.5cm}
\paragraph{APA Citation} \

\begin{hangparas}{.25in}{1}
Mulase, M., \& Penkava, M. (2012, March). \textit{Topological recursion for the Poincaré polynomial of the combinatorial moduli space of curves}. Redirecting. https://doi.org/10.1016/j.aim.2012.03.027 
\end{hangparas}

\vspace*{-0.5cm}
\paragraph{Notes} \

\begin{itemize}
    \item Euler characteristic of moduli space of smooth algebraic curves with distinct marked points.
    \item Harer-Zagier's closed formula, Penner's relation to quantum field theory.
    \item Ribbon graphs in Penner model, isomorphism of topological orbifolds.
    \item Penner model as the generating function, matrix integral 
    \item Introduction to Eynard-Orantin topological recursion theory.
    \item Definition of Poincaré polynomial and its uniqueness (Theorem 1.1).
    \item Symmetric differential and its relation to Eynard-Orantin theory.
    \item Topological recursion's inductive structure and its role in various areas.
    \item Virasoro Constraint and Differential Equation
    \item Derivation of differential recursion equation for Poincaré polynomial (Theorem 5.1).
    \item Initial values calculation (Section 6) and consequences of the differential equation.
    \item Proof that Poincaré polynomial is a Laurent polynomial (Theorem 7.2).
    \item Invariance under the coordinate change (Proposition 7.4).
\end{itemize}
    
\vspace*{-0.5cm}
\paragraph{Questions this source helped me answer} \

\begin{itemize}
    \item How is the Poincaré polynomial related to the Euler characteristic of the moduli space of smooth algebraic curves with marked points?
    \item What is the significance of ribbon graphs in the context of the Penner model and quantum field theory?
    \item How does topological recursion theory contribute to the understanding of the Poincaré polynomial?
    \item What is the differential equation satisfied by the Poincaré polynomial, and how does it connect to the Virasoro constraint condition?
    \item In what way is the Poincaré polynomial related to the intersection numbers of the Deligne-Mumford stack?
\end{itemize}

\vspace*{-0.5cm}
\paragraph{Questions I have for the author} \ 

\begin{itemize}
    \item Can you elaborate on the motivation behind using Eynard-Orantin topological recursion theory in this context?
    \item How does the Laurent polynomial property of the Poincaré polynomial impact its applications and interpretations?
    \item Could you provide more insights into the coordination change invariance of the Poincaré polynomial?
\end{itemize}

\vspace{-0.5cm}
\paragraph{Vocabulary} \ 

\begin{itemize}
\item Moduli space: A mathematical space that represents the collection of all geometric objects with a certain property.
\item Ribbon graph: A graph used in mathematical studies, particularly in the context of algebraic curves.
\item Topological recursion theory: A mathematical theory used to compute certain topological invariants.
\item Virasoro constraint condition: A condition related to the Virasoro algebra, often encountered in string theory and mathematical physics.
\item Laurent polynomial: A polynomial with both positive and negative degree terms, allowing for negative powers of the variable.
\end{itemize}

\vspace*{-0.5cm}
\paragraph{Summary} \

The paper explores the Poincaré polynomial's connection to the Euler characteristic of moduli spaces, specifically focusing on smooth algebraic curves with marked points. It introduces the use of ribbon graphs in the Penner model and establishes the Poincaré polynomial through topological recursion theory. The paper provides a differential equation for the Poincaré polynomial, connecting it to the Virasoro constraint condition and showcasing its Laurent polynomial properties.

\vspace*{-0.5cm}
\paragraph{Implications for my project/thinking} \

This source deepens my understanding of the mathematical intricacies involved in studying algebraic curves and moduli spaces. The use of topological recursion theory and the connection to the Virasoro constraint condition could inspire new perspectives or methodologies in the topological model. Additionally, the Laurent polynomial properties suggest potential applications or considerations for polynomial structures in related contexts.

\newpage 

\research{Dobromir Iliev Iliev}{Research Paper}
\vspace*{-0.5cm}
\paragraph{APA Citation} \

\begin{hangparas}{.25in}{1}
Mitin, I., Korotaev, R., Ermolaev, A., Mironov, V., Lobov, S. A., \& Kazantsev, V. B. (2022, November 28). \textit{Bioinspired propulsion system for a thunniform robotic fish}. MDPI. https://doi.org/10.3390/biomimetics7040215
\end{hangparas}

\vspace*{-0.5cm}
\paragraph{Notes} \

\begin{itemize}
    \item The robot's maximum speed was approximately 0.4 body lengths per second (BL/s), achieved at the maximal tail beat frequency of 3.4 Hz and a maximal tail deflection amplitude of 105 mm.
    \item The speed of the robot was found to increase with the frequency of tail fin oscillations. Different frequencies showed qualitatively similar profiles, shifting to higher speeds as the tail beat frequency increased.
    \item The dependencies of the robot's speed on the amplitude of tail beats exhibited non-monotonic shapes with seemingly oscillatory patterns. Two linear trends with different slopes were observed for small and larger amplitude values, intersecting at an amplitude of about 50 mm.
    \item The oscillatory character of the speed-amplitude dependencies may result from a resonant interaction of the robot's oscillating body and hydrodynamic waves reflected from the pool walls.
    \item The cost of transport (COT), a measure of energy efficiency, showed a minimum corresponding to the change in the angle of the regression lines in the speed-amplitude dependencies. COT values were higher compared to live tuna but varied in an interval of 30-70 J kg1m-1 for different values of frequency and swimming speed.
    \item Increasing the tail oscillation amplitude above a threshold of about 50 mm led to a decrease in swimming efficiency, indicating that there is an energetically preferable range of traveling speeds up to a threshold speed.
    \item Generally, increasing the swimming speed is preferable by increasing the oscillation frequency of the caudal fin rather than the amplitude.    
\end{itemize}
    
\vspace*{-0.5cm}
\paragraph{Questions this source helped me answer} \

\begin{itemize}
    \item How does the thunniform swimming robot's speed vary with tail beat frequency and amplitude?
    \item What is the influence of dynamic parameters on the efficiency of the robotic fish's propulsion system?
    \item What is the cost of transport (COT) for the thunniform swimming robot, and how does it compare to live tuna?
    \item What are the advantages and limitations of the simplified design of the robotic fish's tail section?
    \item How do the experimental results contribute to the understanding of biomimetic robotic fish design?   
\end{itemize}

\vspace*{-0.5cm}
\paragraph{Questions I have for the author} \ 

\begin{itemize}
    \item Can you elaborate on the reasons behind the non-monotonic shapes and seemingly oscillatory patterns observed in the dependencies of the robot's speed on tail beat frequency and amplitude?
    \item How might the resonant interaction of the robot's oscillating body and hydrodynamic waves affect its performance, and are there potential ways to mitigate this effect?
    \item Could you provide insights into potential applications of the thunniform swimming robot and areas for future research in this field? 
\end{itemize}

\vspace{-0.5cm}
\paragraph{Vocabulary} \ 

\begin{itemize}
\item Thunniform: Relating to a type of locomotion characterized by undulation limited to the rear part of the body, as observed in certain fishes like tunas.
\item Flexor/Extensor: Muscles responsible for bending (flexor) and straightening (extensor) a joint or body part.
\item Hydrodynamics: The study of fluid motion and the forces acting on solid bodies immersed in fluids.
\item Resonant Interaction: The phenomenon where an oscillating system is influenced by external forces or frequencies that match its natural frequency, resulting in increased amplitude or efficiency.
\item Cost of Transport (COT): A measure of the energy required to move a unit mass over a unit distance at a given speed.
\end{itemize}

\vspace*{-0.5cm}
\paragraph{Summary} \

The article discusses experiments with a thunniform swimming robot designed to emulate the movement of tuna fish. The robot's propulsion system involves an elastic cord with flexor/extensor mechanisms simulating muscular contractions. The experiments revealed that the robot's speed increased with tail beat frequency and exhibited non-monotonic patterns with amplitude. The cost of transport (energy efficiency) showed variations, and the simplified design of the robotic fish's tail section influenced its speed and efficiency. The study emphasizes the need for dynamic feedback systems to compensate for physical fluctuations and suggests future research directions.

\vspace*{-0.5cm}
\paragraph{Implications for my project/thinking} \

Understanding the impact of tail beat frequency and amplitude on the robotic fish's performance can inform the design of biomimetic propulsion systems in aquatic robots. The insights into energy efficiency and the trade-off between speed and amplitude provide valuable considerations for optimizing the performance of robotic fish.
The acknowledgment of variability in experimental results and the need for dynamic adjustments align with the challenges in developing responsive and adaptive control systems for aquatic robots in real-world environments.